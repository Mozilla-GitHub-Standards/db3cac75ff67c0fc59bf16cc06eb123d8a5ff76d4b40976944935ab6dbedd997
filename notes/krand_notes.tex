%%%%%%%%%%%%%%%%%%%%%%%%%%%%%%%%%%%%%%%%%%%%%%%%%%%%%%%%%%%%%%%%%%%%%%%%%%%%%
%% 
%% Informal notes on k-randomization.
%% 
%%%%%%%%%%%%%%%%%%%%%%%%%%%%%%%%%%%%%%%%%%%%%%%%%%%%%%%%%%%%%%%%%%%%%%%%%%%%%


\documentclass[11pt,draft]{article}
%\documentclass[11pt,draft]{amsart}

% Custom styling.
\usepackage{mozdp}
%% Controls enumeration labels
%\usepackage{enumerate}
%% Shrinks margins 
\usepackage{fullpage}
%% Shows equation label keys
%\usepackage[notref]{showkeys}

%% Title matter
\title{K-Randomization}
\author{Maxim Zhilyaev \and David Zeber}


%%%%%%%%%%%%%%%%%%%%%%%%%%%%%%%%%%%%%%%%%%%%%%%%%%%%%%%%%%%%%%%%%%%%%%%%%%%%%

\begin{document}
\maketitle

\section{Differential Privacy}

The typical setting for differential privacy is the following. 
We consider a \textbf{database} as a collection of records. 
The records are elements of some space $D$, and a database $\boldsymbol{x}$ is a vector of $n$ records: $\boldsymbol{x} \in D^n$.

We wish to release information based on the database by applying a \textbf{query} to it.
This is a function $A$ mapping the database into another space: 
$\map{A}{D^n}{\boldsymbol{S}}$. 
If the function $A$ is random, \ie $A(\boldsymbol{x}) = A(\boldsymbol{x}, X)$ for a random element $X$, then the output $A(\boldsymbol{x})$ is a random element of $\boldsymbol{S}$.

In considering the differential privacy of $A$, we compare the result of applying $A$ to two very similar databases $\boldsymbol{x},\ \boldsymbol{x'}
\in D^n$.
We say the databases \textbf{differ in one row} if 
$\sum_{i=1}^n \ind{x_i \neq x'_i} = 1$.
The random query $A$ is said to be $\epsilon$-\textbf{differentially private} if, for any two databases $\boldsymbol{x},\ \boldsymbol{x'} \in D^n$ differing in
one row,
\[ P[A(\boldsymbol{x}) \in S] \leq \epsilon \cdot P[A(\boldsymbol{x'}) \in S] \]
for all $S \cont \boldsymbol{S}$ (measurable).
An alternative notion of differing in one row that is sometimes used is that
$\boldsymbol{x} \in D^n$, $\boldsymbol{x'} \in D^{n+1}$,
and $x_i = x'_i$ for $i = 1,\dots,n$.
In other words, $\boldsymbol{x'}$ includes an additional record that is not in $\boldsymbol{x}$.

If $\boldsymbol{S}$ is countable, then we can write 
\[ P[A(\boldsymbol{x}) \in S] = \sum_{s \in S} P[A(\boldsymbol{x}) = s]. \]
Hence, 
\[ \frac{P[A(\boldsymbol{x}) \in S]}{P[A(\boldsymbol{x'}) \in S]} = 
    \frac{\sum_{s \in S} P[A(\boldsymbol{x}) = s]}
    {\sum_{s \in S} P[A(\boldsymbol{x'}) = s]} \leq
\max_{s \in S} \frac{P[A(\boldsymbol{x}) = s]}{P[A(\boldsymbol{x'}) = s]} \]
by the Lemma (need reference).

Furthermore, if $A$ randomizes each record in the database independently,
\ie $A(\boldsymbol{x}) = A(\boldsymbol{x}, \boldsymbol{X}) := 
\big(A_0(x_1, X_1),\dots,A_0(x_n,X_n)\big)$
where $X_i$ are independent,
then $\boldsymbol{S} = \boldsymbol{S}_0^n$ and $s = (s_1,\dots,s_n)$ with 
$s_i \in \boldsymbol{S}_0$. 
In this case
$P[A(\boldsymbol{x}) = s] = P[A_0(x_1) = s_1,\dots,A_0(x_n) = s_n]
= \prod P[A_0(x_i) = s_i]$.
If $\boldsymbol{x}$ and $\boldsymbol{x'}$ differ in one row (wlog $x_1 \neq x'_1$ and $x_i = x'_i$ for $i = 2,\dots,n$),
then
\[ \frac{P[A(\boldsymbol{x}) = s]}{P[A(\boldsymbol{x'}) = s]} = 
\frac{P[A_0(x_1) = s_1]}{P[A_0(x'_1) = s_1]}. \]
Therefore, in this case, the query $A$ will satisfy differential privacy if
\[ P[A_0(x) = s] \leq \epsilon \cdot P[A_0(x') = s] \]
for all $x,x' \in D$ and $s \in \boldsymbol{S}_0$.
This is the formulation used in the RAPPOR paper that applies to differences between individual records rather than collections differing on a single element.


-------------------------------------------------------

Consider a collection $T$ of bit vectors, and write $T_v = T\backslash\{v\}$.
The randomization procedure $R$ is $\epsilon$-differentially private if
\[ \log\left(\frac{P[R(T)\in S]}{P[R(T_v)\in S]}\right) \leq \epsilon \]
for any set of bit vectors $S$.

Anonymity:
\[
    A_p = \min_{v \in T, s \in \{0,1\}^L} \frac{P[s \in R(T_v)]}{P[s = R(v)]}
\]

\section{Theoretical setup}

In the following we work with data in the form of bit vectors.
A \textbf{bit vector} is a vector $v \in \{0,1\}^L$.

First we define the randomization procedure we will be applying.

\begin{defn}
The randomization procedure $R$ with \textbf{lie probability} $0 < q < 1/2$ flips a bit with probability $q$, and leaves it as-is with probability $1-q$.
In other words, for a bit $b \in \{0,1\}$,
\[ R(b) = R(b; X) = (1-b)\cdot X + b \cdot (1-X) \quad\text{where } X\sim Ber(q). \]
When applied to a vector, each bit is randomized independently:
\[ R(v) = R\big(v; (X_1,\dots,X_L)\big) = 
    \big(R(v_1; X_1),\dots,R(v_L; X_L)\big) 
\quad\text{where } X \iid Ber(q). \]
\end{defn}

\begin{rmk}
The randomization $R$ reports the original bit value with probability $1-q > q$, and lies with probability $q$.
This is equivalent to the randomized response procedure where the value is reported as-is with probability $1-f$, and with probability $f$ the reported value is the outcome of the toss of a fair coin.
In this case, $q = f/2$.
\end{rmk}
\begin{rmk}
If $q = 1/2$, then $R(0)\eqdist R(1)$, and the reported value is ``completely'' randomly generated, \ie independently of the original value.
\end{rmk}

Distribution of $R(v)$.

For a bit $b$, the randomization lies iff $R(b) \neq b$: 
\[
    P[R(b) = s] = q^{\ind{b \neq s}}(1-q)^{\ind{b = s}}
\]
Hence, for a bit vector $v$, 
\[ P[R(v) = s] = q^{\sum\ind{b_i \neq s_i}}(1-q)^{\sum\ind{b_i = s_i}} 
    = q^{L-m(v,s)}(1-q)^{m(v,s)}, \]
where $m(v,s) = |\{i: v_i = s_i\}|$. 
Note that this probability is maximized when $m(v,s) = L$ (the reported vector $s$ is identical to the original vector $v$), and minimized when $m(v,s) = 0$.
In other words, the most likely outcome of randomizing a bit vector is obtaining an identical vector.

For a collection $T$,
\[ 
    P[s \in R(T)] = 1-P[s \not\in R(T)] = 1-\prod_{v \in T} P[R(v) \neq s]
    = 1-\prod_{v \in T} \big[1 - q^{L-m(v,s)}(1-q)^{m(v,s)} \big].
\]

\end{document}















