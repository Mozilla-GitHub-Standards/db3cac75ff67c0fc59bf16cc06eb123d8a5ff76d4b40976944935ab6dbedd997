%%%%%%%%%%%%%%%%%%%%%%%%%%%%%%%%%%%%%%%%%%%%%%%%%%%%%%%%%%%%%%%%%%%%%%%%%%%%%
%% 
%% Informal notes on k-randomization.
%% 
%%%%%%%%%%%%%%%%%%%%%%%%%%%%%%%%%%%%%%%%%%%%%%%%%%%%%%%%%%%%%%%%%%%%%%%%%%%%%


\documentclass[11pt,draft]{article}
%\documentclass[11pt,draft]{amsart}

% Custom styling.
\usepackage{mozdp}
%% Controls enumeration labels
%\usepackage{enumerate}
%% Shrinks margins 
\usepackage{fullpage}
%% Shows equation label keys
%\usepackage[notref]{showkeys}
%% Title matter
\title{}
\author{Maxim Zhilyaev \and David Zeber}


%%%%%%%%%%%%%%%%%%%%%%%%%%%%%%%%%%%%%%%%%%%%%%%%%%%%%%%%%%%%%%%%%%%%%%%%%%%%%

\begin{document}
\maketitle

\section{recursion formular - simple version}

Consider generating function for Poisson-Binomial of you case.  $m$ Bernoulli trials with success $p$ and $n$ Bernoulli trials with success $q$. 

\begin{align}
m+n = N \\
q+p = 1 \\
G(x) = (q+px)^m \cdot (p+qx)^n
\end{align}

The derivative of $ln(f(x)$ is given by:

\begin{align}
[ln(g(x))]' = \frac{g^`(x)}{f(x)} = \frac{\left ( \sum_{i=0}^{N} a_i x^i \right )^`}{ \sum_{i=0}^{N} a_i x^i} = \frac{\sum_{i=1}^{N} i \cdot a_i x^{i-1}}{  \sum_{i=0}^{N} a_i x^i}
\end{align}

On the other hand
\begin{align}
[ln(g(x_)]^` = (m(q+px) + n(p+qx))^` = \frac{mp}{q+px} + \frac{nq}{p+qx} = \frac{xpq(m+n) + mp^2 + nq^2}{x^2pq + x(p^2 + q^2) + pq}
\end{align}

Equating both expressions we get
\begin{align}
\frac{xpqN + mp^2 + nq^2}{x^2pq + x(p^2 + q^2) + pq} = \frac{\sum_{i=1}^{N} i \cdot a_i x^{i-1}}{  \sum_{i=0}^{N} a_i x^i} \\
(xpqN + mp^2 + nq^2)(\sum_{i=0}^{N} a_i x^i) = (x^2pq + x(p^2 + q^2) + pq)(\sum_{i=1}^{N} i \cdot a_i x^{i-1})
\end{align}

Multiplying and equating terms with same power of $x$ we get:
\begin{align}
a_i(mp^2 + nq^2) + a_{i-1}pqN = a_{i+1}pq(i+1) + a_i(p^2 + q^2)i + a_{i-1}pq(i-1) \\
a_i(mp^2 + nq^2) + a_{i-1}pqN = a_{i+1}pqi + a_i(p^2 + q^2)i + a_{i-1}pqi  + (a_{i+1} - a_{i-1})
\end{align}

Ignore the difference of $a_{i+1} - a_{i-1}$, and denote the expectation of successes as $\mu$. Then the expression simplifies to:
\begin{align}
\frac{   N - i}{ \frac{ a_{i+1}}{a_i}i    - (\frac{\mu - Npq}{pq} - \frac{p^2 + q^2}{pq}i)   } =  \frac{a_{i}}{a_{i-1}} \\
\frac{   N - i}{ N - i  + (\frac{ a_{i+1}}{a_i} - 1)i   - \frac{\mu - i}{pq}   } =  \frac{a_{i}}{a_{i-1}} \\
\end{align}

Denote the distance between $i$ and $\mu$ as $l$.  Then:
\begin{align}
\mu - i = l\\
i = \mu -l \\
\frac{a_{i+1}}{a_{i}} = f_l \\
\frac{a_{i}}{a_{i-1}} = f_{l+1} \\
\frac{   N - \mu + l}{ N - \mu + l  + (f_l - 1)(\mu-l)  - \frac{l}{pq}   } =  f_{l+1}  \\
 f_{l+1} =  \frac{  1 }{ 1  - \frac{1}{N-\mu + l}(\frac{l}{pq} - (f_l - 1)(\mu - l)) } 
\end{align}

Differentiating by $\mu$, we have:
\begin{align}
F'(\mu) = -\dfrac{pq\left(\left(f_l-1\right)Npq-l\right)}{\left(\left(f_l-2\right)pqx+\left(n+\left(2-f_l\right)l\right)pq-l\right)^2}
\end{align}

Since the denominator is always positive, the derivative is always negative as long as for all $l$ the inequality below holds:
\begin{align}
(f_l-1)Npq > l
\end{align}

This is a magic formula.  No matter what I do, and how I twist the proof, it always comes down to that inequality.
I tested it empirically and it indeed works, and in fact $(f_l-1)Npq$ grows much much faster than $l$.
But i had difficulties proving that analytically.
A very interesting observation that $Npq$ is the variance, which would suggest using Chebusheff inequality, but this, again, didn't get me anywhere.
I know the inequality holds, but not sure how to prove it.

Note that when $l=0$  and $f_0=1$, $f_1 = 1$ for all $\mu$.  When $l=1$, $f_1$ is given by
\begin{align}
f_{1} =  \frac{  1 }{ 1  - \frac{1}{N-\mu + 1}\frac{1}{pq} } 
\end{align}

Clearly,  $f_1$ is largest for the smallest $\mu$ which is reached when $m=0$ and the smallest when $m=N$.
And we also know that $f_l$ when $m=0$ is always greater than $f_l$ for $m=N$, perhaps there's some way to utilize monotonicity of the $f_{l+1}$ when $f_l$ is fixed.


\end{document}



