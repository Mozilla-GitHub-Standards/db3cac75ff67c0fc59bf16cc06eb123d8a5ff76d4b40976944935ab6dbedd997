%%%%%%%%%%%%%%%%%%%%%%%%%%%%%%%%%%%%%%%%%%%%%%%%%%%%%%%%%%%%%%%%%%%%%%%%%%%%%
%% 
%% Informal notes on k-randomization.
%% 
%%%%%%%%%%%%%%%%%%%%%%%%%%%%%%%%%%%%%%%%%%%%%%%%%%%%%%%%%%%%%%%%%%%%%%%%%%%%%


\documentclass[11pt,draft]{article}
%\documentclass[11pt,draft]{amsart}

% Custom styling.
\usepackage{mozdp}
%% Controls enumeration labels
%\usepackage{enumerate}
%% Shrinks margins 
\usepackage{fullpage}
%% Shows equation label keys
%\usepackage[notref]{showkeys}
%% Title matter
\title{}
\author{Maxim Zhilyaev \and David Zeber}


%%%%%%%%%%%%%%%%%%%%%%%%%%%%%%%%%%%%%%%%%%%%%%%%%%%%%%%%%%%%%%%%%%%%%%%%%%%%%

\begin{document}
\maketitle

\section{recursion formular - simple version}

Consider generating function for Poisson-Binomial of you case.  $m$ Bernoulli trials with success $p$ and $n$ Bernoulli trials with success $q$. 

\begin{align}
m+n = N \\
q+p = 1 \\
G(x) = (q+px)^m \cdot (p+qx)^n
\end{align}

The derivative of $ln(a(x)$ is given by:

\begin{align}
[ln(g(x)]' = \frac{g'(x)}{g(x)} = \frac{\left ( \sum_{i=0}^{N} a_i x^i \right )^`}{ \sum_{i=0}^{N} a_i x^i} = \frac{\sum_{i=1}^{N} i \cdot a_i x^{i-1}}{  \sum_{i=0}^{N} a_i x^i}
\end{align}

On the other hand
\begin{align}
[ln(g(x))]' = (m(q+px) + n(p+qx))^` = \frac{mp}{q+px} + \frac{nq}{p+qx} = \frac{xpq(m+n) + mp^2 + nq^2}{x^2pq + x(p^2 + q^2) + pq}
\end{align}

Equating both expressions we get
\begin{align}
\frac{xpqN + mp^2 + nq^2}{x^2pq + x(p^2 + q^2) + pq} = \frac{\sum_{i=1}^{N} i \cdot a_i x^{i-1}}{  \sum_{i=0}^{N} a_i x^i} \\
(xpqN + mp^2 + nq^2)(\sum_{i=0}^{N} a_i x^i) = (x^2pq + x(p^2 + q^2) + pq)(\sum_{i=1}^{N} i \cdot a_i x^{i-1})
\end{align}

Multiplying and equating terms with same power of $x$ we get:
\begin{align}
a_i(mp^2 + nq^2) + a_{i-1}pqN = a_{i+1}pq(i+1) + a_i(p^2 + q^2)i + a_{i-1}pq(i-1) \\
a_i(mp^2 + nq^2) + a_{i-1}pqN = a_{i+1}pqi + a_i(p^2 + q^2)i + a_{i-1}pqi  + (a_{i+1} - a_{i-1})
\end{align}

Ignore the difference of $a_{i+1} - a_{i-1}$, and denote the expectation of successes as $\mu$. Then the expression simplifies to:
\begin{align}
\frac{   N - i}{ \frac{ a_{i+1}}{a_i}i    - (\frac{\mu - Npq}{pq} - \frac{p^2 + q^2}{pq}i)   } =  \frac{a_{i}}{a_{i-1}} \\
\frac{   N - i}{ N - i  + (\frac{ a_{i+1}}{a_i} - 1)i   - \frac{\mu - i}{pq}   } =  \frac{a_{i}}{a_{i-1}} \\
\end{align}

Denote the distance between $i$ and $\mu$ as $l$.  Then:
\begin{align}
\mu - i = l\\
i = \mu -l \\
\frac{a_{i+1}}{a_{i}} = f_l \\
\frac{a_{i}}{a_{i-1}} = f_{l+1} \\
\frac{   N - \mu + l}{ N - \mu + l  + (f_l - 1)(\mu-l)  - \frac{l}{pq}   } =  f_{l+1}  \\
 f_{l+1} =  \frac{  1 }{ 1  - \frac{1}{N-\mu + l}(\frac{l}{pq} - (f_l - 1)(\mu - l)) } 
\end{align}

Note that when $l=0$  and $f_0=1$, $f_1 = 1$ for all $\mu$.  When $l=1$, $f_2$ is given by
\begin{align}
f_{2} =  \frac{  1 }{ 1  - \frac{1}{N-\mu + 1}\frac{1}{pq} } 
\end{align}

Clearly,  $f_2$ is largest for the smallest $\mu$ which is reached when $m=0$ and the smallest when $m=N$.
Denote $\mu_0$ and $\mu_x$ as expectations at $m=0$ and $m=x$ respectively.  Obviously  $\mu_0 < \mu_x$.
Suppose that for some $l+1$,  the corresponding ratio $f_l$ of distribution with $\mu_x$ becomes larger then that of distribution with $\mu_0$.
\begin{align}
f^x_{l+1} > f^0_{l+1} \\
\end{align}

Using the recursion formula, we can express  $f^x_{l+1}$ and  $f^0_{l+1}$ as:
\begin{align}
f^x_{l+1} = \frac{  1 }{ 1  - \frac{1}{N-\mu_x + l}(\frac{l}{pq} - (f^x_l - 1)(\mu_x - l)) } \\
f^0_{l+1} = \frac{  1 }{ 1  - \frac{1}{N-\mu_0 + l}(\frac{l}{pq} - (f^0_l - 1)(\mu_0 - l)) }
\end{align}

For $f^x_{l+1}  > f^0_{l+1}$ to hold, the following must hold:

\begin{align}
\frac{1}{N-\mu_x + l}(\frac{l}{pq} - (f^x_l - 1)(\mu_x - l)) >  \frac{1}{N-\mu_0 + l}(\frac{l}{pq} - (f^0_l - 1)(\mu_0 - l))  \\
\frac{(f^0_l - 1)(\mu_0 - l)}{N-\mu_0 + l} - \frac{(f^x_l - 1)(\mu_x - l)}{N-\mu_x + l} > \frac{1}{N-\mu_0 + l} \frac{l}{pq}  - \frac{1}{N-\mu_x + l}\frac{l}{pq}   \\
\frac{(f^0_l - 1)(\mu_0 - l)}{N-\mu_0 + l} - \frac{(f^x_l - 1)(\mu_x - l)}{N-\mu_x + l} > - \frac{l}{pq} \frac{\mu_x - \mu_0}{((N-\mu_x + l)(N-\mu_0 + l)} \\
\frac{(f^x_l - 1)(\mu_x - l)}{N-\mu_x + l}  - \frac{(f^0_l - 1)(\mu_0 - l)}{N-\mu_0 + l} < \frac{l}{pq} \frac{\mu_x - \mu_0}{((N-\mu_x + l)(N-\mu_0 + l)}
\end{align}

Suppose $f^x_l \ge f^0_l$, then we can replace $f^x_l - 1$ with $f^0_l - 1$ and the inequality should still hold since we reduce the left part.

\begin{align}
\frac{(f^x_0 - 1)(\mu_x - l)}{N-\mu_x + l}  - \frac{(f^0_l - 1)(\mu_0 - l)}{N-\mu_0 + l} < \frac{l}{pq} \frac{\mu_x - \mu_0}{(N-\mu_x + l)(N-\mu_0 + l)} \\
(f^0_l - 1) \left [ \frac{\mu_x - l}{N-\mu_x + l}  - \frac{\mu_0 - l}{N-\mu_0 + l} \right ] < \frac{l}{pq} \frac{\mu_x - \mu_0}{(N-\mu_x + l)(N-\mu_0 + l)}
\end{align}

The expression inside the brackets simplifies to:
\begin{align}
\frac{\mu_x - l}{N-\mu_x + l}  - \frac{\mu_0 - l}{N-\mu_0 + l} = \frac{N(\mu_x-\mu_0)}{(N-\mu_x + l)(N-\mu_0 + l)}
\end{align}

From here:
\begin{align}
(f^0_l - 1) \frac{N(\mu_x-\mu_0)}{(N-\mu_x + l)(N-\mu_0 + l)} < \frac{l}{pq} \frac{\mu_x - \mu_0}{(N-\mu_x + l)(N-\mu_0 + l)} \\
(f^0_l - 1)Npq < l \\
(f^0_l - 1) < \frac{l}{Npq}
\end{align}

Now consider the values of probabilistic ratio for $\mu_0$.  Since $m=0$, all Bernoulli trials generate successes with probability $q$, hence the ratio at any given number of successes $i$ is:
\begin{align}
\frac{a_i}{a_{i-1}} = \frac{\binom{N}{i} q^ip^{N-i}}{\binom{N}{i-1} q^{i-1}p^{N-i+1}} = \frac{N-i+1}{i}\frac{q}{p} \\
f^0_l = \frac{N-\mu_0 + l+1}{\mu_0 - l}\frac{q}{p} = \frac{N- qN+ l+1}{qN - l}\frac{q}{p}  \\
f^0_l  - 1 =  \frac{N- qN+ l+1}{qN - l}\frac{q}{p} - 1 = \frac{(N- qN+ l+1)q - (qN-l)p}{(qN-l)p} = \\
\frac{Nq -q^2N +lq+q - pqN +pl}{(qN-l)p} = \frac{Nq(1-p) -q^2N +l(q+p)+q}{(qN-l)p} = \\
 \frac{Nq^2 -q^2N +l+q}{(qN-l)p} = \frac{l+q}{(qN-l)p}  \\
 f^0_l  - 1 =  \frac{l+q}{(qN-l)p}
\end{align}

Now, consider the inequality:
\begin{align}
 f^0_l  - 1 =  \frac{l+q}{(qN-l)p} >  \frac{l}{Npq} \\
 \frac{l+q}{qN-l} >  \frac{l}{qN} 
\end{align}

Since this inequality holds for all $l$, we arrived at contradiction.   Then $f_x$ must be strictly less than $f_0$.   
That's interesting, for it means that if  $f^x_l > f^0_l$ and iteration $l$, then at iteration $l+1$,     $f^0_l > f^x_l$.
Basically, $f^0_l$ may only loose it maximum status for a single iteration.
This could be sufficient for our purpose, but i am still looking for a better prove. 
 
\end{document}



